\documentclass[times,specification,annotation]{itmo-student-thesis}
\usepackage{fancyhdr}
\usepackage{amsmath}
\usepackage{makecell}
\newcommand{\TheOnlyTruePageStyle}{plain}

%% Опции пакета:
%% - specification - если есть, генерируется задание, иначе не генерируется
%% - annotation - если есть, генерируется аннотация, иначе не генерируется
%% - times - делает все шрифтом Times New Roman, собирается с помощью xelatex
%% - languages={...} - устанавливает перечень используемых языков. По умолчанию это {english,russian}.
%%                     Последний из языков определяет текст основного документа.

%% Делает запятую в формулах более интеллектуальной, например:
%% $1,5x$ будет читаться как полтора икса, а не один запятая пять иксов.
%% Однако если написать $1, 5x$, то все будет как прежде.
\usepackage{icomma}

\usepackage{subfiles}
%% Один из пакетов, позволяющий делать таблицы на всю ширину текста.
\usepackage{tabularx}

%% Данные пакеты необязательны к использованию в бакалаврских/магистерских
%% Они нужны для иллюстративных целей
%% Начало
\usepackage{tikz}
\usetikzlibrary{arrows}
\usepackage{filecontents}
%% Конец

%% Указываем файл с библиографией.
\addbibresource{bachelor-thesis.bib}

\begin{document}


%% Эта команда генерирует титульный лист и аннотацию.
% 
\studygroup{R34353}
\title{Разработка методов генерации видео для анализа однородных данных в промышленности}
\author{Золотарев Дмитрий Валерьевич}{Золотарев Д.В.}
\supervisor{Ефимова Валерия Александровна}{Ефимова В.А.}{кандидат т.н.}{научный сотрудник Университета ИТМО}
\publishyear{2025}
%% Дата выдачи задания. Можно не указывать, тогда надо будет заполнить от руки.
\startdate{01}{февраля}{2025}
%% Срок сдачи студентом работы. Можно не указывать, тогда надо будет заполнить от руки.
\finishdate{15}{мая}{2025}
%% Дата защиты. Можно не указывать, тогда надо будет заполнить от руки.
\defencedate{15}{июня}{2019}

\addconsultant{Румянцева М.Ю.}{без степени, без звания}

\secretary{Рукуйжа Е.В.}

%% Задание
%%% Техническое задание и исходные данные к работе
\technicalspec{Требуется разработать стилевой файл для системы \LaTeX, позволяющий оформлять бакалаврские работы и магистерские диссертации
на кафедре компьютерных технологий Университета ИТМО. Стилевой файл должен генерировать титульную страницу пояснительной записки,
задание, аннотацию и содержательную часть пояснительной записк. Первые три документа должны максимально близко соответствовать шаблонам документов,
принятым в настоящий момент на кафедре, в то время как содержательная часть должна максимально близко соответствовать ГОСТ~7.0.11-2011
на диссертацию.}

%%% Содержание выпускной квалификационной работы (перечень подлежащих разработке вопросов)
\plannedcontents{Пояснительная записка должна демонстрировать использование наиболее типичных конструкций, возникающих при составлении
пояснительной записки (перечисления, рисунки, таблицы, листинги, псевдокод), при этом должна быть составлена так, что демонстрируется
корректность работы стилевого файла. В частности, записка должна содержать не менее двух приложений (для демонстрации нумерации рисунков и таблиц
по приложениям согласно ГОСТ) и не менее десяти элементов нумерованного перечисления первого уровня вложенности (для демонстрации корректности
используемого при нумерации набора русских букв).}

%%% Исходные материалы и пособия 
\plannedsources{\begin{enumerate}
    \item ГОСТ~7.0.11-2011 <<Диссертация и автореферат диссертации>>;
    \item С.М. Львовский. Набор и верстка в системе \LaTeX;
    \item предыдущий комплект стилевых файлов, использовавшийся на кафедре компьютерных технологий.
\end{enumerate}}

%%% Цель исследования
\researchaim{Разработка удобного стилевого файла \LaTeX
             для бакалавров и магистров кафедры компьютерных технологий.}

%%% Задачи, решаемые в ВКР
\researchtargets{\begin{enumerate}
    \item обеспечение соответствия титульной страницы, задания и аннотации шаблонам, принятым в настоящее время на кафедре;
    \item обеспечение соответствия содержательной части пояснительной записки требованиям ГОСТ~7.0.11-2011 <<Диссертация и автореферат диссертации>>;
    \item обеспечение относительного удобства в использовании~--- указание данных об авторе и научном руководителе один раз и в одном месте, автоматический подсчет числа тех или иных источников.
\end{enumerate}}

%%% Использование современных пакетов компьютерных программ и технологий
\addadvancedsoftware{Пакет \texttt{tabularx} для чуть более продвинутых таблиц}{\ref{sec:tables}, Приложения~\ref{sec:app:1}, \ref{sec:app:2}}
\addadvancedsoftware{Пакет \texttt{biblatex} и программное средство \texttt{biber}}{Список использованных источников}

%%% Краткая характеристика полученных результатов 
\researchsummary{Получился, надо сказать, практически неплохой стилевик. В 2015--2018 годах
его уже использовали некоторые бакалавры и магистры. Надеюсь на продолжение.}

%%% Гранты, полученные при выполнении работы 
\researchfunding{Автор разрабатывал этот стилевик исключительно за свой счет и на
добровольных началах. Однако значительная его часть была бы невозможна, если бы
автор не написал в свое время кандидатскую диссертацию в \LaTeX,
а также не отвечал за формирование кучи научно-технических отчетов по гранту,
известному как <<5-в-100>>, что происходило при государственной финансовой поддержке
ведущих университетов Российской Федерации (субсидия 074-U01).}

%%% Наличие публикаций и выступлений на конференциях по теме выпускной работы
\researchpublications{По теме этой работы я (к счастью!) ничего не публиковал.
\begin{refsection}
Однако покажу, как можно ссылаться на свои публикации из списка литературы:
\nocite{example-english, example-russian}
\printannobibliography
\end{refsection}
}

\maketitle{Бакалавр}


%% Оглавление
\tableofcontents

%% Макрос для введения. Совместим со старым стилевиком.
\subfile{sections/introduction}

% %% Начало содержательной части.
% ===========================================================================
% ===========================================================================
\subfile{sections/domain_analysis}


% ===========================================================================
% ===========================================================================
\subfile{sections/solutions_analysis}


% ===========================================================================
% ===========================================================================
\subfile{sections/research}


% ===========================================================================
% ===========================================================================

\subfile{sections/generation}



% %% Так помечается начало обзора.
% \startrelatedwork
% Пример ссылок в рамках обзора: \cite{example-english, example-russian, unrestricted-jump-evco, doerr-doerr-lambda-lambda-self-adjustment-arxiv}.
% %% Так помечается конец обзора.
% \finishrelatedwork
% Вне обзора:~\cite{bellman}.

% \section{Таблицы}\label{sec:tables}

% В качестве примера таблицы приведена таблица~\ref{tab1}.

% \begin{table}[!h]
% \caption{Таблица умножения (фрагмент)}\label{tab1}
% \centering
% \begin{tabular}{|*{18}{c|}}\hline
% -- & 1 & 2 & 3 & 4 & 5 & 6 & 7 & 8 & 9 & 10 & 11 & 12 & 13 & 14 & 15 & 16 & 17 \\\hline
% 1  & 1 & 2 & 3 & 4 & 5 & 6 & 7 & 8 & 9 & 10 & 11 & 12 & 13 & 14 & 15 & 16 & 17 \\\hline
% 2  & 2 & 4 & 6 & 8 & 10 & 12 & 14 & 16 & 18 & 20 & 22 & 24 & 26 & 28 & 30 & 32 & 34 \\\hline
% 3  & 3 & 6 & 9 & 12 & 15 & 18 & 21 & 24 & 27 & 30 & 33 & 36 & 39 & 42 & 45 & 48 & 51 \\\hline
% 4  & 4 & 8 & 12 & 16 & 20 & 24 & 28 & 32 & 36 & 40 & 44 & 48 & 52 & 56 & 60 & 64 & 68 \\\hline
% \end{tabular}
% \end{table}

% Есть еще такое окружение \texttt{tabularx}, его можно аккуратно растянуть на всю страницу.
% Приведем пример (таблица~\ref{tab2}).

% \begin{table}[!h]
% \caption{Таблица умножения с помощью \texttt{tabularx} (фрагмент)}\label{tab2}
% \centering
% \begin{tabularx}{\textwidth}{|*{18}{>{\centering\arraybackslash}X|}}\hline
% -- & 1 & 2 & 3 & 4 & 5 & 6 & 7 & 8 & 9 & 10 & 11 & 12 & 13 & 14 & 15 & 16 & 17 \\\hline
% 1  & 1 & 2 & 3 & 4 & 5 & 6 & 7 & 8 & 9 & 10 & 11 & 12 & 13 & 14 & 15 & 16 & 17 \\\hline
% 2  & 2 & 4 & 6 & 8 & 10 & 12 & 14 & 16 & 18 & 20 & 22 & 24 & 26 & 28 & 30 & 32 & 34 \\\hline
% 3  & 3 & 6 & 9 & 12 & 15 & 18 & 21 & 24 & 27 & 30 & 33 & 36 & 39 & 42 & 45 & 48 & 51 \\\hline
% 4  & 4 & 8 & 12 & 16 & 20 & 24 & 28 & 32 & 36 & 40 & 44 & 48 & 52 & 56 & 60 & 64 & 68 \\\hline
% \end{tabularx}
% \end{table}

% \section{Рисунки}

% Пример рисунка (c помощью \texttt{TikZ}) приведен на рисунке~\ref{fig1}. Под \texttt{pdflatex} можно также
% использовать \texttt{*.jpg}, \texttt{*.png} и даже \texttt{*.pdf}, под \texttt{latex} можно использовать
% Metapost. Последний можно использовать и под \texttt{pdflatex}, для чего в стилевике продекларированы
% номера картинок от~1 до~20.

% \begin{figure}[!h]
% \caption{Пример рисунка}\label{fig1}
% \centering
% \begin{tikzpicture}[scale=0.7]
% \draw[thick,->] (0,0)--(3.5,0);
% \draw[thick,->] (0,0)--(0,3.5);
% \draw[very thick, red] (0,0)--(3,3);
% \draw[dashed] (3,0)--(3,3);
% \draw[dashed] (1.5,0)--(1.5,1.5);
% \end{tikzpicture}
% \end{figure}

% \section{Листинги}

% В работах студентов кафедры <<Компьютерные технологии>> часто встречаются листинги. Листинги бывают
% двух основных видов~--- исходный код и псевдокод. Первый оформляется с помощью окружения \texttt{lstlisting}
% из пакета \texttt{listings}, который уже включается в стилевике и немного настроен. Пример Hello World на Java
% приведен на листинге~\ref{lst1}. Пример большого листинга~--- в приложении (листинг~\ref{lstX}).

% \begin{lstlisting}[float=!h,caption={Пример исходного кода на Java},label={lst1}]
% public class HelloWorld {
%     public static void main(String[] args) {
%         System.out.println("Hello, world!");
%     }
% }
% \end{lstlisting}

% Псевдокод можно оформлять с помощью разных пакетов. В данном стилевике включается пакет \texttt{algorithmicx}.
% Сам по себе он не генерирует флоатов, поэтому для них используется пакет \texttt{algorithm}.
% Пример их совместного использования приведен на листинге~\ref{lst2}.

% \begin{algorithm}[!h]
% \caption{Пример псевдокода}\label{lst2}
% \begin{algorithmic}
% 	\Function{IsPrime}{$N$}
% 		\For{$t \gets [2; \lfloor\sqrt{N}\rfloor]$}
% 			\If{$N \bmod t = 0$}
% 				\State\Return \textsc{false}
% 			\EndIf
% 		\EndFor
% 		\State\Return \textsc{true}
% 	\EndFunction
% \end{algorithmic}
% \end{algorithm}

% Наконец, листинги из \texttt{listings} тоже можно подвешивать с помощью \texttt{algorithm},
% пример на листинге~\ref{lst3}.

% \begin{algorithm}[!h]
% \caption{Исходный код и флоат \texttt{algorithm}}\label{lst3}
% \begin{lstlisting}
% public class HelloWorld {
%     public static void main(String[] args) {
%         System.out.println("Hello, world!");
%     }
% }
% \end{lstlisting}
% \end{algorithm}

% \chapter{Проверка сквозной нумерации}

% Листинг~\ref{lst4} должен иметь номер 4.

% \begin{algorithm}[!h]
% \caption{Исходный код и флоат \texttt{algorithm}}\label{lst4}
% \begin{lstlisting}
% public class HelloWorld {
%     public static void main(String[] args) {
%         System.out.println("Hello, world!");
%     }
% }
% \end{lstlisting}
% \end{algorithm}

% Рисунок~\ref{fig2} должен иметь номер 2.

% \begin{figure}[!h]
% \caption{Пример рисунка}\label{fig2}
% \centering
% \begin{tikzpicture}[scale=0.7]
% \draw[thick,->] (0,0)--(3.5,0);
% \draw[thick,->] (0,0)--(0,3.5);
% \draw[very thick, red] (0,0)--(3,3);
% \draw[dashed] (3,0)--(3,3);
% \draw[dashed] (1.5,0)--(1.5,1.5);
% \end{tikzpicture}
% \end{figure}

% Таблица~\ref{tab3} должна иметь номер 3.

% \begin{table}[!h]
% \caption{Таблица умножения с помощью \texttt{tabularx} (фрагмент)}\label{tab3}
% \centering
% \begin{tabularx}{\textwidth}{|*{18}{>{\centering\arraybackslash}X|}}\hline
% -- & 1 & 2 & 3 & 4 & 5 & 6 & 7 & 8 & 9 & 10 & 11 & 12 & 13 & 14 & 15 & 16 & 17 \\\hline
% 1  & 1 & 2 & 3 & 4 & 5 & 6 & 7 & 8 & 9 & 10 & 11 & 12 & 13 & 14 & 15 & 16 & 17 \\\hline
% 2  & 2 & 4 & 6 & 8 & 10 & 12 & 14 & 16 & 18 & 20 & 22 & 24 & 26 & 28 & 30 & 32 & 34 \\\hline
% 3  & 3 & 6 & 9 & 12 & 15 & 18 & 21 & 24 & 27 & 30 & 33 & 36 & 39 & 42 & 45 & 48 & 51 \\\hline
% 4  & 4 & 8 & 12 & 16 & 20 & 24 & 28 & 32 & 36 & 40 & 44 & 48 & 52 & 56 & 60 & 64 & 68 \\\hline
% \end{tabularx}
% \end{table}

% % В конце каждой главы желательно делать выводы. Вывод по данной главе~--- нумерация работает корректно, ура!

% %% Макрос для заключения. Совместим со старым стилевиком.
\startconclusionpage

% В данном разделе размещается заключение.

\printmainbibliography

% %% После этой команды chapter будет генерировать приложения, нумерованные русскими буквами.
% %% \startappendices из старого стилевика будет делать то же самое
\appendix

\chapter{}\label{sec:app:1}
\chapter{}\label{sec:app:2}

\end{document}