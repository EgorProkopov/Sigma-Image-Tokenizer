\section*{Заключение}
\addcontentsline{toc}{section}{Заключение}

В результате выполненного исследования были разработаны и оценены новые методы токенизации изображений на основе матричных разложений, которые решают ряд ограничений классичесского подхода токенизации, предложенного в Vision Transformer. Поставленные цели работы были достигнуты. Предложенный методы токенизации сокращают длину входной последовательности без существенной потери качества модели. Также разработаны адаптивные методы отбора токенов входной последовательности, что позволяет работать с изображениями различного разрешения.

Предложенные методы токенизации были протестированы при решении задач многоклассовой классификации, регрессии, генерации изображений методом автокодировщика и авторегрессионным способом. 


