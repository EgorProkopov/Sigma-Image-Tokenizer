\documentclass[times,specification,annotation]{itmo-student-thesis}
\usepackage{fancyhdr}

%% Опции пакета:
%% - specification - если есть, генерируется задание, иначе не генерируется
%% - annotation - если есть, генерируется аннотация, иначе не генерируется
%% - times - делает все шрифтом Times New Roman, собирается с помощью xelatex
%% - languages={...} - устанавливает перечень используемых языков. По умолчанию это {english,russian}.
%%                     Последний из языков определяет текст основного документа.

%% Делает запятую в формулах более интеллектуальной, например:
%% $1,5x$ будет читаться как полтора икса, а не один запятая пять иксов.
%% Однако если написать $1, 5x$, то все будет как прежде.
\usepackage{icomma}

%% Один из пакетов, позволяющий делать таблицы на всю ширину текста.
\usepackage{tabularx}

%% Данные пакеты необязательны к использованию в бакалаврских/магистерских
%% Они нужны для иллюстративных целей
%% Начало
\usepackage{tikz}
\usetikzlibrary{arrows}
\usepackage{filecontents}
%% Конец

%% Указываем файл с библиографией.
\addbibresource{bachelor-thesis.bib}


\begin{document}

\chapter{Использование предложенных алгоритмов токенизации изображений для решения различных нейросетевых задач обработки изображений}

\section{Решение задачи регрессии}

\subsection{Решение задачи регрессии моделью с токенизатором, основанным на быстром преобразовании Фурье}


\subsection{Решение задачи регрессии моделью с токенизатором mSVD}


\section{Генерация изображений с использованием предложенных методов токенизации}

\subsection{Использование алгоритма токенизации, основанного на быстром преобразовании Фурье}


\subsection{Использование алгоритма токенизации mSVD}

% Старый текст


Детокенизатор mSVD строится как отраженная копия токенизатора mSVD. В начале, набор векторов размерности

$$
\Big(\dfrac{M^2}{16^2}, s_E\Big)
$$

преобразуется в тензор размерности

$$
\Big(s_E, \dfrac{M}{16}, \dfrac{M}{16}\Big)
$$

Следом, полученный тензор обрабатывается последовательность алгоритмов пиксельных перемешиваний, функций активации и свёрточных слоёв c размером ядра свёртки 3, шагом 1, отступом 1 и с нейроном смещения. Однако, поскольку в mSVD токенизаторе присутстовало две ветви нейронных сетей, ответственных за получение наборов векторов $u$ и $v$ соответственно, то для поддержания симметрии токенизатора и детокенизатора, вместо одного свёрточного слоя на один этап пиксельного перемешивания, используется 2 свёрточных слоя.

Таким образом, детокенизатор mSVD имеет 4 блока, состоящих из свёрточного слоя, нормализации, функции активации, свёрточного слоя, нормализации, функции активации и пиксельного перемешивания.

Полученный после преобразования тензор имеет размерность 

$$
\Big(\dfrac{s_E}{16^2},  M, M\Big)
$$

Данный тензор проецируется в трехканальное изображение с помощью слоя проекции. Данный слой представлен в виде свёрточного слоя с размером ядра свёртки 1, шагом свёртки 1, размером отступа 0 и без нейрона смещения.






\subsection{Авторегрессионная генерация изображений с использованием mSVD-токенизатора}

Для решения задачи авторегрессионной генерации вдохновимся идеями, предложенными в работе DALLE. Так, представим токенизатор как модель-кодировщик, задача которой - закодировать изображение в набор векторов. В работе, посвященной DALLE, используется дискретный вариационный автокодировщик. Поскольку трансформер работает только с наборами векторов, то после обработки закодированного набора векторов трансформером, для восстановления изображения потребуется детокенизатор.


%\subsection{Обучение токенизатора и детокинезатора для авторегрессионной генерации изображений}


%\subsection{Обучение трансформера для авторегрессионной генерации изображений с использованием токинезатора и детокинезатора mSVD}

% \chapterconclusion

\end{document}